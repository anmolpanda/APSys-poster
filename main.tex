%%%%%%%%%%%%%%%%%%%%%%%%%%%%%%%%%%%%%%%%%
% Jacobs Landscape Poster
% LaTeX Template
% Version 1.0 (29/03/13)
%
% Created by:
% Computational Physics and Biophysics Group, Jacobs University
% https://teamwork.jacobs-university.de:8443/confluence/display/CoPandBiG/LaTeX+Poster
% 
% Further modified by:
% Nathaniel Johnston (nathaniel@njohnston.ca)
%
% This template has been downloaded from:
% http://www.LaTeXTemplates.com
%
% 
% Masaryk University presentation themes were downloaded from:
% https://www.overleaf.com/gallery/tagged/muni
%
% and ported into Jacobs Landscape Poster by:
% Jumaidil Awal (ideal1st.here@googlemail.com)
% 
% Jacobs Landscape Poster License:
% CC BY-NC-SA 3.0 (http://creativecommons.org/licenses/by-nc-sa/3.0/)
%
% Masaryk University's fibeamer theme license:
% Copyright 2015  Vít Novotný <witiko@mail.muni.cz>
% Faculty of Informatics, Masaryk University (Brno, Czech Republic)
% under Latex Project Public License
%
%%%%%%%%%%%%%%%%%%%%%%%%%%%%%%%%%%%%%%%%%

%----------------------------------------------------------------------------------------
%	PACKAGES AND OTHER DOCUMENT CONFIGURATIONS
%----------------------------------------------------------------------------------------

\documentclass[final]{beamer}

\usepackage[scale=1.24]{beamerposter} % Use the beamerposter package for laying out the poster

%\usetheme{confposter} % Use the confposter theme supplied with this template
\usetheme[faculty=chemo]{fibeamer} % Uncomment to use Masaryk University's fibeamer theme instead.

%\setbeamercolor{block title}{fg=ngreen,bg=white} % Colors of the block titles
%\setbeamercolor{block body}{fg=black,bg=white} % Colors of the body of blocks
%\setbeamercolor{block alerted title}{fg=white,bg=dblue!70} % Colors of the highlighted block titles
%\setbeamercolor{block alerted body}{fg=black,bg=dblue!10} % Colors of the body of highlighted blocks
% Many more colors are available for use in beamerthemeconfposter.sty

%-----------------------------------------------------------
% Define the column widths and overall poster size
% To set effective sepwid, onecolwid and twocolwid values, first choose how many columns you want and how much separation you want between columns
% In this template, the separation width chosen is 0.024 of the paper width and a 4-column layout
% onecolwid should therefore be (1-(# of columns+1)*sepwid)/# of columns e.g. (1-(4+1)*0.024)/4 = 0.22
% Set twocolwid to be (2*onecolwid)+sepwid = 0.464
% Set threecolwid to be (3*onecolwid)+2*sepwid = 0.708

\newlength{\sepwid}
\newlength{\onecolwid}
\newlength{\twocolwid}
\newlength{\threecolwid}
\setlength{\paperwidth}{46.8in} % A0 width: 46.8in
\setlength{\paperheight}{33.1in} % A0 height: 33.1in
\setlength{\sepwid}{0.024\paperwidth} % Separation width (white space) between columns
\setlength{\onecolwid}{0.21\paperwidth} % Width of one column
\setlength{\twocolwid}{0.451\paperwidth} % Width of two columns
\setlength{\threecolwid}{0.678\paperwidth} % Width of three columns
%\setlength{\topmargin}{-0.5in} % Reduce the top margin size
%-----------------------------------------------------------

\usepackage{graphicx}  % Required for including images

\usepackage{booktabs} % Top and bottom rules for tables

\usepackage{multicol} % For multicols

\usepackage{mdframed} % for border around figures

\usepackage{mathabx} % for up and down arrow

\usepackage{listings} % for C code

\usepackage{threeparttable} % for caption alignment in optimizations
%----------------------------------------------------------------------------------------
%	TITLE SECTION 
%----------------------------------------------------------------------------------------

\title{A Preliminary Performance Model for Optimizing Software Packet Processing Pipelines} % Poster title
\author{Ankit Bhardwaj, Atul Shree, Bhargav Reddy V and Sorav Bansal} % Author(s)
\institute{Department of Computer Science and Engineering, Indian Institute of Technology Delhi} % Institution(s)

%----------------------------------------------------------------------------------------

\begin{document}
\addtobeamertemplate{block end}{}{\vspace*{2ex}} % White space under blocks
\addtobeamertemplate{block example end}{}{\vspace*{2ex}} % White space under example blocks
\addtobeamertemplate{block alerted end}{}{\vspace*{2ex}} % White space under highlighted (alert) blocks

\setlength{\belowcaptionskip}{2ex} % White space under figures
\setlength\belowdisplayshortskip{2ex} % White space under equations
%\begin{darkframes} % Uncomment for dark theme, don't forget to \end{darkframes}
\begin{frame} % The whole poster is enclosed in one beamer frame

%==========================Begin Head===============================
  \begin{columns}
   \begin{column}{\linewidth}
    \vskip1cm
    \centering
    \usebeamercolor{title in headline}{\color{fg}\Huge{\textbf{\inserttitle}}\\[0.5ex]}
    \usebeamercolor{author in headline}{\color{fg}\Large{\insertauthor}\\[1ex]}
    \usebeamercolor{institute in headline}{\color{fg}\large{\insertinstitute}\\[1ex]}
    \vskip1cm
   \end{column}
   \vspace{1cm}
  \end{columns}
 \vspace{1cm}

%==========================End Head===============================

\begin{columns}[t] % The whole poster consists of three major columns, the second of which is split into two columns twice - the [t] option aligns each column's content to the top

\begin{column}{\sepwid}\end{column} % Empty spacer column

\begin{column}{\onecolwid} % The first column

%----------------------------------------------------------------------------------------
%	OBJECTIVES
%----------------------------------------------------------------------------------------

\begin{exampleblock}{Motivation and Objectives}
\begin{itemize}
 \item Newer functionalities are being developed over time with increasing popularity of SDN.
 \item Researchers are developing DSLs to make the task easier. However, ease of programming doesn't guarantee application performance.
\end{itemize}
\vspace{1cm}
\textbf{Goal}: Develop a compiler that can automatically map a high-level specification to an underlying machine architecture in an optimal way.
\end{exampleblock}

%----------------------------------------------------------------------------------------
%	Compilation
%----------------------------------------------------------------------------------------

\begin{exampleblock}{Compilation Phases}


\begin{figure}
\fbox{\includegraphics[width=0.8\linewidth]{img/compilation_phases}}
\caption{Compilation phases}
\end{figure}
Applications written in P4\cite{p4} are given as the input to P4C\cite{p4c} compiler and the compiler generated DPDK\cite{DPDK} based Applications. DPDK\cite{DPDK} application is compiled using gcc to generate the target binary.
\end{exampleblock}

%----------------------------------------------------------------------------------------
%	Piplining
%----------------------------------------------------------------------------------------

\begin{exampleblock}{Packet Processing Pipeline}
\begin{figure}
\fbox{\includegraphics[width=0.8\linewidth]{img/packet_process_pipeline}}
\caption{Packet Processing Pipeline}
\end{figure}
\begin{enumerate}[*]
 \item Exploit DMA bandwidth between NIC and Main Memory labeled \textcircled{1} in Figure
 \item Exploit Memory Level Parallelism between CPU and Memory labeled \textcircled{2} in Figure
\end{enumerate}
\end{exampleblock}


%------------------------------------------------
%	TRANSFORMATIONS
%------------------------------------------------

\end{column} % End of the first column
\begin{column}{\sepwid}\end{column} % Empty spacer column
\begin{column}{\twocolwid} % Begin a column which is two columns wide (column 2)
\begin{exampleblock}{Compiler Transformations related to Batching, Sub-Batching and Prefetching}
\begin{multicols}{3}
%----------------------------------------------
%	BATCHING
%----------------------------------------------
\begin{figure}[ht]
\begin{tiny}
\begin{tabular}[b]{p{\onecolwid}}
sub app \{ \\
\hspace{0.2\sepwid}for (i = 0; i < B; i++) \{ \\
\hspace{0.4\sepwid}p = read\_from\_input\_NIC(); \\
\hspace{0.4\sepwid}p = process\_packet(p); \\
\hspace{0.4\sepwid}write\_to\_output\_NIC(p); \\
\hspace{0.2\sepwid}\}\\
\vspace{1cm}
\hspace{1\sepwid}$\updownarrows$\\
\vspace{1cm}
sub app \{\\
\hspace{0.2\sepwid}for (i = 0; i < B; i++)\\
\hspace{0.4\sepwid}p[i] = read\_from\_input\_NIC();\\
\hspace{0.2\sepwid}for (i = 0; i < B; i++)\\
\hspace{0.4\sepwid}p[i] = process\_packet(p[i]);\\
\hspace{0.2\sepwid}for (i = 0; i < B; i++)\\
\hspace{0.4\sepwid}write\_to\_output\_NIC(p[i]);\\
\}
\end{tabular}
\end{tiny}
% \caption{Batching}
\end{figure}
Loop Fission\cite{loop_fission} Transformation for Batching
%----------------------------------------------
%	SUB-BATCHING
%----------------------------------------------
\begin{figure}[ht]
\begin{tiny}
\begin{tabular}[b]{p{\onecolwid}}
sub{ process\_packet}(p) \{ \\
\hspace{0.2\sepwid}for (i = 0; i < B; i++) \{ \\
\hspace{0.4\sepwid}t1 = lookup\_table1(p[i]); \\
\hspace{0.4\sepwid}t2 = lookup\_table2(p[i], t1); \\
\hspace{0.4\sepwid}\ldots\\
\hspace{0.2\sepwid}\}\hspace{0.2\sepwid}\}\\
\vspace{1cm}
\hspace{1\sepwid}$\updownarrows$\\
\vspace{1cm}
sub {process\_packet}(p) \{\\
\hspace{0.2\sepwid}for (i = 0; i < B; i+=b) \{\\
\hspace{0.4\sepwid}for (j = i; j < i+b; j++)\\
\hspace{0.6\sepwid}t1[j-i] = lookup\_table1(p[j]);\\
\hspace{0.4\sepwid}for (j = i; j < i+b; j++)\\
\hspace{0.6\sepwid}t1[j-i] = lookup\_table2(p[j], t1[j-i]);\\
\hspace{0.4\sepwid}\ldots\\
\hspace{0.2\sepwid}\}\hspace{0.2\sepwid}\}
\end{tabular}
\end{tiny}
% \caption{Sub-batching}
\end{figure}

Loop Fission\cite{loop_fission} Transformation for Sub-Batching
%----------------------------------------------
%	PREFETCHING
%----------------------------------------------
\begin{figure}[ht]
\begin{tiny}
\begin{tabular}[b]{p{\onecolwid}}

sub\hspace{0.2\sepwid} hash\_lookup() \{\\
\hspace{0.4\sepwid}for( i=0; i<B; i++)\{\\
\hspace{0.6\sepwid}key\_hash[i] = hash\_compute(key[i]);\\
\hspace{0.6\sepwid}prefetch(bucket(key-hash[i]));\\
\hspace{0.4\sepwid}\}\\
\hspace{0.4\sepwid}for( i=0; i < B; i++)\{\\
\hspace{0.6\sepwid}val[j] = hash\_lookup(key\_hash[j]);\\
\}\hspace{0.4\sepwid}\}

\hspace{1\sepwid}$\updownarrows$\\
 
sub{ hash\_lookup()} \{\\
\hspace{0.4\sepwid}for( i = 0; i < B; i += b)\{\\
\hspace{0.6\sepwid}for( j = i; j < i + b; j++ )\{\\
\hspace{0.8\sepwid}key\_hash[i] = hash\_compute(key[i]);\\
\hspace{0.8\sepwid}prefetch(bucket(key-hash[i]));\\
\hspace{0.6\sepwid}\}\\
\hspace{0.6\sepwid}for( j = i; j < i + b; j++ )\{\\
\hspace{0.8\sepwid}val[j] = hash\_lookup(key\_hash[j]);\\
\}\hspace{0.4\sepwid}\}\hspace{0.2\sepwid}\}
\end{tabular}
\end{tiny}
\end{figure}
Use of Sub-Batching for Prefetching
\end{multicols}
\end{exampleblock}
%----------------------------------------------------------------------------------------
%	IMPORTANT RESULT
%----------------------------------------------------------------------------------------

\begin{alertblock}{Performance Model}
\begin{itemize}
 \item Let the service rate of CPU, CPU-Memory, I/O-Memory interface be {\bf c, m, d}.
 \item {\bf Throughput} of the application will be {\bf min(c, m, d)}.
 \item The value of {\bf m} will vary with change in {\bf b} and will follow some $\boldsymbol{f_{mem}(m, b)}$.
 \item DMA interface is not linearly scalable and {\bf d} increases based on some function $\boldsymbol{f_{dma} (d,B)}$.
 \item $\boldsymbol{Throughput = min(c, f_{mem}(m, b), f_{dma})(d, B))}$
\end{itemize}  
\end{alertblock} 

%----------------------------------------------------------------------------------------

\begin{columns}[t,totalwidth=\twocolwid] % Split up the two columns wide column again
\begin{column}{\twocolwid} % The first column within column 2 (column 2.1)
\begin{exampleblock}{Experiments and Results}
\begin{multicols}{3}

\begin{figure}
\fbox{\includegraphics[width=0.99\linewidth]{img/batching_and_prefetching}}
\caption{Effect of Batching and Prefetching}
\end{figure}

\begin{figure}
\fbox{\includegraphics[width=0.99\linewidth]{img/throughput_sensitivity_B}}
\caption{Sensitivity of Throughput to B. Solid line b = B and dotted lines b = 1.}
\end{figure}

\begin{figure}
\fbox{\includegraphics[width=0.99\linewidth]{img/throughput_sensitivity_b}}
\caption{Sensitivity of Throughput to b. Batch Size B = 128 for these experiments.}
\end{figure}

\end{multicols}
\end{exampleblock}

%----------------------------------------------------------------------------------------

\end{column} % End of column 2.1
\begin{column}{\sepwid}\end{column} % Empty spacer column
\end{columns} % End of the split of column 2

\end{column} % End of the second column

\begin{column}{\sepwid}\end{column} % Empty spacer column

% ---------------------------------------------------------------------------------------
% Column 3
% ---------------------------------------------------------------------------------------

\begin{column}{\onecolwid} % The third column

%----------------------------------------------------------------------------------------
%	CONCLUSION
%----------------------------------------------------------------------------------------

\begin{exampleblock}{Discussion}
\begin{itemize}
 \item Use of Batching({\bf B}) to exploit NIC-Memory Parallelism.
 \item Use of Prefetching to exploit MLP at Memory-CPU interface in the pipeline.
 \item Use of sub-batch({\bf b}) to adjust the prefetch distance.
 \item Run the applications to find optimal value of {\bf b} and {\bf B}.
 \item Input {\bf b} and {\bf B} to the compiler to generate optimized DPDK application.
\end{itemize}
\end{exampleblock}

%----------------------------------------------------------------------------------------
%	ADDITIONAL INFORMATION
%----------------------------------------------------------------------------------------

\begin{exampleblock}{Future Work}
\begin{itemize}
\item Get more understanding of NIC-Memory interface and CPU-Memory interface. 
\item Make the model robust to be used for various kinds of applications.
\item Extend the model for heterogeneous architectures.
\end{itemize}
\end{exampleblock}

%----------------------------------------------------------------------------------------
%	REFERENCES
%----------------------------------------------------------------------------------------

\begin{exampleblock}{References}

\nocite{*} % Insert publications even if they are not cited in the poster
\small{\bibliographystyle{unsrt}
\bibliography{sample}\vspace{1cm}}
\end{exampleblock}


\begin{center}
\includegraphics[width=0.6\linewidth]{img/iitd_logo.jpg}
\end{center}

%----------------------------------------------------------------------------------------

\end{column} % End of the third column

\begin{column}{\sepwid}\end{column} % Empty spacer column

\end{columns} % End of all the columns in the poster

\end{frame} % End of the enclosing frame
%\end{darkframes} % Uncomment for dark theme
\end{document}
